%需使用xelatex编译
\documentclass[UTF8,a4paper]{ctexart}
\usepackage[colorlinks,linkcolor=blue]{hyperref}
\usepackage{enumitem}
\usepackage{amssymb}
\usepackage{amsmath}
\usepackage{subfigure}
%\usepackage{fancyhdr}
\usepackage[left=2.5cm,right=2.5cm,top=2.5cm,bottom=2.5cm]{geometry}%页边距
\usepackage{titlesec}
\usepackage{listings}
\usepackage[table]{xcolor}
\usepackage{graphicx}
\usepackage{array}% 表格
\usepackage{longtable}%% 长表格
\usepackage{pdfpages}
\usepackage{bm}
\usepackage[section]{placeins}%禁止浮动体跨subsection
\linespread{1}%单倍行距
%附录代码格式
\newfontfamily\consolas{Consolas}
\lstset{
columns=fullflexiblem,
breaklines=true,
%numbers=left,%在左侧显示行号
frame=none,%不显示背景边框
backgroundcolor=\color[RGB]{245,245,244},%设定背景颜色
%keywordstyle=\color[RGB]{40,40,255},%设定关键字颜色
%numberstyle=\consolas\color{darkgray},%设定行号格式
commentstyle=\tt,%设置代码注释的格式
%stringstyle=\color{myorange},%设置字符串格式
showstringspaces=false,%不显示字符串中的空格
basicstyle=\consolas,%\color{mygreen},
extendedchars=false 
}
\numberwithin{equation}{section}%新节编号自动清0
\numberwithin{table}{section}
\numberwithin{figure}{section}
\renewcommand{\theequation}{\arabic{section}.\arabic{equation}}%公式按节编号
\renewcommand{\thefigure}{\arabic{section}.\arabic{figure}}
\renewcommand{\thetable}{\arabic{section}.\arabic{table}}
\begin{document}
%\begin{titlepage}
%\includepdf{firstpage.pdf}%封面
%\end{titlepage}
\pagestyle{plain}
\begin{center}
\bfseries\zihao{3}
鲈鱼质量估计模型
\end{center}


\begin{center}
\bfseries\zihao{4}
摘要
\end{center}
\zihao{-4}%
本文对鲈鱼质量与身长,胸围的关系进行了建模,得到了三个鲈鱼质量估计模型。采用模型一对已给数据进行最小二乘拟合的到质量估计式,拟合优度为$R^2=$.
对于模型二,质量估计式为,拟合优度$R^2$,模型三质量估计式式,拟合优度。其中模型对数据的拟合效果最好。
\CTEXsetup[name={,、},number={\chinese{section}}]{section}%一级标题序号中文
\titleformat*{\section}{\centering\zihao{4}\bfseries}%一级标题字体
\titleformat{\subsection}{\zihao{-4}\bfseries}{\thesubsection}{1em}{}%二级标题字体
\titleformat{\subsubsection}{\zihao{-4}\bfseries}{\thesubsubsection}{1em}{}
\zihao{-4}%正文字号
\newpage
\section{问题重述}

\section{模型假设}
\begin{enumerate}[label=(\arabic*)]
\item 假设1
\item 假设2
\end{enumerate}
\section{符号说明}
%\renewcommand\arraystretch{2}%宽松
%\begin{center}
%\begin{tabular}{|c|c|}

%\end{center}
\section{问题分析}
%插入图片
%\begin{figure}[!htp]
%	\centering
%	\includegraphics[width=7cm]{Pictures/图像坐标系.pdf}
%	\caption{图像坐标系}
%	\label{fig:tx}
%\end{figure}


%\FloatBarrier%禁止上面的浮动体跨过这里
%插入子图
%\begin{figure}[!htp]\setcounter{subfigure}{0}
%	\centering
%	\subfigure[T=0]{\includegraphics[width=4.5cm]{Pictures/b0..png}}
%	\subfigure[T=0.2]{\includegraphics[width=4.5cm]{Pictures/b0.2.png}}
%	\subfigure[T=0.4]{\includegraphics[width=4.5cm]{Pictures/b0.4.png}}
%	\subfigure[T=0.6]{\includegraphics[width=4.5cm]{Pictures/b0.6.png}}
%	\subfigure[T=0.8]{\includegraphics[width=4.5cm]{Pictures/b0.8.png}}
%	\subfigure[T=0.9]{\includegraphics[width=4.5cm]{Pictures/b0.9.png}}
%	\caption{几种阈值下的a01的二值图}
%	\label{fig:yzt}
%\end{figure}
%引用子图
%\ref{fig:ds}\subref{ds1}

%\begin{thebibliography}{99}
%参考文献
 % \bibitem{b8}FITS Working Group,Definition of the Flexible Image Transport System \url{https://fits.gsfc.nasa.gov/standard40/fits_standard40draft1.pdf},2017.5.13
%\end{thebibliography}
\section*{附录}
%\noindent{\heiti\zihao{-4}附录一:Mathematica 代码}
%\lstinputlisting[language=Mathematica]{../CodeAndData/mma.txt}

%\noindent{\heiti\zihao{-4}附录二:Matlab 代码}
%\lstinputlisting[language=Matlab]{../CodeAndData/p1.m}
%\noindent{\heiti\zihao{-4}附录一:Mathematica 代码}
%\centerline{M1.nb}
%\lstinputlisting[language=Mathematica]{../支撑文件/MathematicaCode.txt}
%\begin{lstlisting}
%\end{lstlisting}
\end{document}
